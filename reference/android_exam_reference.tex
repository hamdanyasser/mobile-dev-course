\documentclass[11pt,a4paper]{book}

% Packages
\usepackage[utf8]{inputenc}
\usepackage[english]{babel}
\usepackage{geometry}
\geometry{a4paper, margin=1in}
\usepackage{hyperref}
\usepackage{cleveref}
\usepackage{listings}
\usepackage{xcolor}
\usepackage{tcolorbox}
\usepackage{enumitem}
\usepackage{graphicx}
\usepackage{fancyhdr}
\usepackage{makeidx}
\usepackage[toc]{glossaries}

% Hyperref setup
\hypersetup{
    colorlinks=true,
    linkcolor=blue,
    filecolor=magenta,
    urlcolor=cyan,
    pdftitle={Android Exam Reference - HotelAppRef},
    pdfauthor={Mobile Dev Course},
    bookmarks=true,
}

% Make index and glossary
\makeindex
\makeglossaries

% Code listing styles
\definecolor{codebg}{rgb}{0.95,0.95,0.95}
\definecolor{codegreen}{rgb}{0,0.6,0}
\definecolor{codegray}{rgb}{0.5,0.5,0.5}
\definecolor{codepurple}{rgb}{0.58,0,0.82}
\definecolor{backcolour}{rgb}{0.95,0.95,0.92}

% Java style
\lstdefinestyle{javastyle}{
    language=Java,
    backgroundcolor=\color{backcolour},
    commentstyle=\color{codegreen},
    keywordstyle=\color{blue}\bfseries,
    numberstyle=\tiny\color{codegray},
    stringstyle=\color{codepurple},
    basicstyle=\ttfamily\footnotesize,
    breakatwhitespace=false,
    breaklines=true,
    captionpos=b,
    keepspaces=true,
    numbers=left,
    numbersep=5pt,
    showspaces=false,
    showstringspaces=false,
    showtabs=false,
    tabsize=2,
    frame=single,
    rulecolor=\color{black}
}

% XML style
\lstdefinestyle{xmlstyle}{
    language=XML,
    backgroundcolor=\color{backcolour},
    commentstyle=\color{codegreen},
    keywordstyle=\color{blue}\bfseries,
    numberstyle=\tiny\color{codegray},
    stringstyle=\color{codepurple},
    basicstyle=\ttfamily\footnotesize,
    breakatwhitespace=false,
    breaklines=true,
    captionpos=b,
    keepspaces=true,
    numbers=left,
    numbersep=5pt,
    showspaces=false,
    showstringspaces=false,
    showtabs=false,
    tabsize=2,
    frame=single,
    morekeywords={android,xmlns,tools,layout_width,layout_height}
}

% Gradle style
\lstdefinestyle{gradlestyle}{
    language=Groovy,
    backgroundcolor=\color{backcolour},
    commentstyle=\color{codegreen},
    keywordstyle=\color{blue}\bfseries,
    numberstyle=\tiny\color{codegray},
    stringstyle=\color{codepurple},
    basicstyle=\ttfamily\footnotesize,
    breakatwhitespace=false,
    breaklines=true,
    captionpos=b,
    keepspaces=true,
    numbers=left,
    numbersep=5pt,
    showspaces=false,
    showstringspaces=false,
    showtabs=false,
    tabsize=2,
    frame=single
}

% Default style
\lstset{style=javastyle}

% Glossary entries
\newglossaryentry{activity}{
    name=Activity,
    description={A single screen with a user interface. The fundamental component of Android apps}
}

\newglossaryentry{intent}{
    name=Intent,
    description={A messaging object used to request an action from another app component}
}

\newglossaryentry{recyclerview}{
    name=RecyclerView,
    description={A flexible view for providing a limited window into a large data set}
}

\newglossaryentry{room}{
    name=Room,
    description={Persistence library that provides an abstraction layer over SQLite}
}

\newglossaryentry{viewbinding}{
    name=ViewBinding,
    description={Feature that generates binding classes for XML layouts, enabling type-safe view access}
}

\newglossaryentry{dao}{
    name=DAO,
    description={Data Access Object - interface defining database operations}
}

\newglossaryentry{manifest}{
    name=Manifest,
    description={AndroidManifest.xml file that describes essential information about the app}
}

% Header/Footer
\pagestyle{fancy}
\fancyhf{}
\rhead{Android Exam Reference}
\lhead{\leftmark}
\rfoot{Page \thepage}

% Title page info
\title{\Huge\textbf{Android Development}\\
\Large Exam Reference Guide\\
\vspace{1cm}
\large HotelAppRef Application}
\author{Mobile Development Course}
\date{\today}

\begin{document}

\maketitle

\tableofcontents

\chapter*{Preface}
\addcontentsline{toc}{chapter}{Preface}

This reference guide is designed for open-book Android development exams. It contains:

\begin{itemize}
    \item Complete code examples from the HotelAppRef application
    \item Annotated explanations of why each pattern works
    \item Quick reference recipes for common tasks
    \item Debugging tips and common gotchas
\end{itemize}

\textbf{How to use during exams:}
\begin{enumerate}
    \item Use the index and table of contents to quickly locate topics
    \item Copy code snippets and adapt them to your needs
    \item Pay attention to the ``WHY'' comments explaining each pattern
    \item Check Appendix A for file path references
\end{enumerate}

\part{Zero to Run: Android Studio 101}

\chapter{Android Studio Setup}
\index{Android Studio!installation}

\section{Installation Steps}

\begin{enumerate}
    \item Download Android Studio from \url{https://developer.android.com/studio}
    \item Run the installer and follow the wizard
    \item Install Android SDK (recommended: API 34)
    \item Configure an Android Virtual Device (AVD)
\end{enumerate}

\section{Opening an Existing Project}

\begin{tcolorbox}[title=Quick Steps]
\begin{enumerate}
    \item File → Open
    \item Navigate to project root (contains \texttt{build.gradle.kts})
    \item Click OK
    \item Wait for Gradle sync (3-5 minutes first time)
\end{enumerate}
\end{tcolorbox}

\section{Running Your App}
\index{running app}

\textbf{Method 1: Using the IDE}
\begin{itemize}
    \item Click green ▶ Run button (Shift+F10)
    \item Select emulator or device
    \item Wait for build and installation
\end{itemize}

\textbf{Method 2: Command Line}
\begin{lstlisting}[style=gradlestyle, caption=Build from terminal, label=lst:gradlebuild]
./gradlew assembleDebug
# APK: app/build/outputs/apk/debug/app-debug.apk
\end{lstlisting}

\section{Android Virtual Device (AVD)}
\index{AVD}

\textbf{Creating an Emulator:}
\begin{enumerate}
    \item Tools → Device Manager
    \item Create Device
    \item Select hardware (Pixel 5 recommended)
    \item Select system image (API 34 recommended)
    \item Finish
\end{enumerate}

\section{Logcat: Reading App Logs}
\index{Logcat}

View → Tool Windows → Logcat

\textbf{Log Levels:}
\begin{itemize}
    \item \textcolor{red}{E (Error)}: Critical errors
    \item \textcolor{orange}{W (Warning)}: Potential issues
    \item \textcolor{blue}{I (Info)}: Informational messages
    \item \textcolor{gray}{D (Debug)}: Debug messages
\end{itemize}

\textbf{Filtering logs:}
\begin{lstlisting}[style=javastyle, caption=Using Log in code]
// path: app/src/main/java/.../MainActivity.java
import android.util.Log;

private static final String TAG = "MainActivity";

@Override
protected void onCreate(Bundle savedInstanceState) {
    super.onCreate(savedInstanceState);
    Log.d(TAG, "onCreate called");
}
\end{lstlisting}

\section{Debugging with Breakpoints}
\index{debugging}

\begin{enumerate}
    \item Click left gutter next to line number to set breakpoint (red dot)
    \item Click Debug button (Shift+F9) instead of Run
    \item App pauses at breakpoint
    \item Inspect variables in Debug window
    \item Use Step Over (F8), Step Into (F7), Resume (F9)
\end{enumerate}

\chapter{Gradle Basics}
\index{Gradle}

\section{Understanding Gradle Files}

Gradle is the build system for Android. Two key files:

\begin{enumerate}
    \item \textbf{build.gradle.kts (Project level)} - Project-wide configuration
    \item \textbf{app/build.gradle.kts (Module level)} - App-specific config
\end{enumerate}

\subsection{Project-level build.gradle.kts}

\begin{lstlisting}[style=gradlestyle, caption=Project build.gradle.kts, label=lst:projectgradle]
// path: build.gradle.kts
plugins {
    alias(libs.plugins.android.application) apply false
}
\end{lstlisting}

\subsection{App-level build.gradle.kts}

\begin{lstlisting}[style=gradlestyle, caption=App build.gradle.kts, label=lst:appgradle]
// path: app/build.gradle.kts
plugins {
    id("com.android.application")
}

android {
    namespace = "com.example.hotelappref"
    compileSdk = 34  // SDK version for compilation

    defaultConfig {
        applicationId = "com.example.hotelappref"  // Unique app ID
        minSdk = 24      // Minimum Android version (7.0)
        targetSdk = 34   // Target Android version (14)
        versionCode = 1
        versionName = "1.0"
    }

    buildFeatures {
        viewBinding = true  // Enable ViewBinding
    }
}

dependencies {
    // AndroidX Core
    implementation("androidx.appcompat:appcompat:1.6.1")
    implementation("com.google.android.material:material:1.11.0")

    // RecyclerView
    implementation("androidx.recyclerview:recyclerview:1.3.2")

    // Room Database
    implementation("androidx.room:room-runtime:2.6.1")
    annotationProcessor("androidx.room:room-compiler:2.6.1")
}
\end{lstlisting}

\section{Common Gradle Tasks}

\begin{lstlisting}[style=gradlestyle, caption=Gradle commands]
./gradlew clean          # Clean build artifacts
./gradlew assembleDebug  # Build debug APK
./gradlew assembleRelease  # Build release APK
./gradlew installDebug   # Install debug APK on device
\end{lstlisting}

\section{Syncing Gradle}

When you modify \texttt{build.gradle.kts}:
\begin{enumerate}
    \item Click "Sync Now" in notification bar
    \item Or: File → Sync Project with Gradle Files
\end{enumerate}

\part{Core Android (Java) Recipes}

\chapter{Activities \& Lifecycle}
\index{Activity!lifecycle}

\section{Activity Lifecycle Overview}

An Activity goes through several states:

\begin{center}
\texttt{onCreate() → onStart() → onResume() → [RUNNING] → onPause() → onStop() → onDestroy()}
\end{center}

\subsection{Key Lifecycle Methods}

\begin{lstlisting}[style=javastyle, caption=Activity lifecycle methods, label=lst:lifecycle]
// path: app/src/main/java/com/example/hotelappref/MainActivity.java
public class MainActivity extends AppCompatActivity {

    @Override
    protected void onCreate(Bundle savedInstanceState) {
        super.onCreate(savedInstanceState);
        // Initialize UI, set content view
        // Called when activity is created
    }

    @Override
    protected void onStart() {
        super.onStart();
        // Activity becomes visible
    }

    @Override
    protected void onResume() {
        super.onResume();
        // Activity starts interacting with user
        // Start animations, acquire resources
    }

    @Override
    protected void onPause() {
        super.onPause();
        // Activity losing focus
        // Save data, pause animations
    }

    @Override
    protected void onStop() {
        super.onStop();
        // Activity no longer visible
        // Release resources
    }

    @Override
    protected void onDestroy() {
        super.onDestroy();
        // Activity being destroyed
        // Final cleanup
    }
}
\end{lstlisting}

\subsection{Saving Instance State}
\index{Activity!state}

Handle configuration changes (rotation):

\begin{lstlisting}[style=javastyle, caption=Saving state, label=lst:savestate]
// path: app/src/main/java/.../MainActivity.java
@Override
protected void onSaveInstanceState(Bundle outState) {
    super.onSaveInstanceState(outState);
    // Save data before activity is destroyed
    outState.putString("key", "value");
    outState.putInt("count", counter);
}

@Override
protected void onCreate(Bundle savedInstanceState) {
    super.onCreate(savedInstanceState);

    if (savedInstanceState != null) {
        // Restore saved data
        String value = savedInstanceState.getString("key");
        int count = savedInstanceState.getInt("count");
    }
}
\end{lstlisting}

\section{Creating an Activity}

\textbf{Step 1:} Create Java class extending \texttt{AppCompatActivity}

\textbf{Step 2:} Create layout XML

\textbf{Step 3:} Register in AndroidManifest.xml

\begin{lstlisting}[style=javastyle, caption=Basic Activity, label=lst:basicactivity]
// path: app/src/main/java/com/example/hotelappref/MainActivity.java
package com.example.hotelappref;

import android.os.Bundle;
import androidx.appcompat.app.AppCompatActivity;

public class MainActivity extends AppCompatActivity {

    @Override
    protected void onCreate(Bundle savedInstanceState) {
        super.onCreate(savedInstanceState);
        setContentView(R.layout.activity_main);
    }
}
\end{lstlisting}

\begin{lstlisting}[style=xmlstyle, caption=Registering in Manifest, label=lst:manifestactivity]
<!-- path: app/src/main/AndroidManifest.xml -->
<activity
    android:name=".MainActivity"
    android:exported="true">
    <intent-filter>
        <action android:name="android.intent.action.MAIN" />
        <category android:name="android.intent.category.LAUNCHER" />
    </intent-filter>
</activity>
\end{lstlisting}

\chapter{Intents: Navigating Between Activities}
\index{Intent}

\section{Explicit Intents}

Navigate to specific activity:

\begin{lstlisting}[style=javastyle, caption=Starting another activity, label=lst:explicitintent]
// path: app/src/main/java/.../MainActivity.java
Intent intent = new Intent(MainActivity.this, AddHotelActivity.class);
startActivity(intent);
\end{lstlisting}

\subsection{Passing Data with Intent}

\begin{lstlisting}[style=javastyle, caption=Passing data, label=lst:intentdata]
// Sender (MainActivity)
Intent intent = new Intent(this, DetailsActivity.class);
intent.putExtra("hotel_name", "Grand Plaza");
intent.putExtra("hotel_id", 123);
startActivity(intent);

// Receiver (DetailsActivity)
@Override
protected void onCreate(Bundle savedInstanceState) {
    super.onCreate(savedInstanceState);

    Intent intent = getIntent();
    String name = intent.getStringExtra("hotel_name");
    int id = intent.getIntExtra("hotel_id", -1);
}
\end{lstlisting}

\subsection{Passing Objects}

Object must implement \texttt{Serializable} or \texttt{Parcelable}:

\begin{lstlisting}[style=javastyle, caption=Passing objects, label=lst:intentobject]
// Hotel class implements Serializable
public class Hotel implements Serializable {
    // ... fields
}

// Sender
Hotel hotel = new Hotel(...);
intent.putExtra("hotel", hotel);

// Receiver
Hotel hotel = (Hotel) getIntent().getSerializableExtra("hotel");
\end{lstlisting}

\section{Activity Results}
\index{Activity!results}

Modern way using \texttt{ActivityResultLauncher}:

\begin{lstlisting}[style=javastyle, caption=Activity results, label=lst:activityresult]
// path: app/src/main/java/.../MainActivity.java
public class MainActivity extends AppCompatActivity {

    private ActivityResultLauncher<Intent> addHotelLauncher;

    @Override
    protected void onCreate(Bundle savedInstanceState) {
        super.onCreate(savedInstanceState);

        // Register launcher
        addHotelLauncher = registerForActivityResult(
            new ActivityResultContracts.StartActivityForResult(),
            result -> {
                if (result.getResultCode() == RESULT_OK) {
                    // Success! Handle result
                    Toast.makeText(this, "Hotel added!", Toast.LENGTH_SHORT).show();
                }
            }
        );

        // Launch activity
        button.setOnClickListener(v -> {
            Intent intent = new Intent(this, AddHotelActivity.class);
            addHotelLauncher.launch(intent);
        });
    }
}

// In AddHotelActivity
private void saveHotel() {
    // ... save logic
    setResult(RESULT_OK);  // Signal success
    finish();  // Close activity
}
\end{lstlisting}

\section{Implicit Intents}
\index{Intent!implicit}

Request actions from other apps:

\begin{lstlisting}[style=javastyle, caption=Implicit intents, label=lst:implicitintent]
// Open URL
Intent browserIntent = new Intent(Intent.ACTION_VIEW, Uri.parse("https://example.com"));
startActivity(browserIntent);

// Make phone call (requires CALL_PHONE permission)
Intent callIntent = new Intent(Intent.ACTION_CALL, Uri.parse("tel:+9611234567"));
startActivity(callIntent);

// Send email
Intent emailIntent = new Intent(Intent.ACTION_SENDTO);
emailIntent.setData(Uri.parse("mailto:test@example.com"));
emailIntent.putExtra(Intent.EXTRA_SUBJECT, "Subject");
emailIntent.putExtra(Intent.EXTRA_TEXT, "Email body");
startActivity(Intent.createChooser(emailIntent, "Send email"));

// Share text
Intent shareIntent = new Intent(Intent.ACTION_SEND);
shareIntent.setType("text/plain");
shareIntent.putExtra(Intent.EXTRA_TEXT, "Check out this hotel!");
startActivity(Intent.createChooser(shareIntent, "Share via"));
\end{lstlisting}

\chapter{Views, Layouts \& Resources}
\index{View}\index{Layout}

\section{Common Views}

\subsection{TextView}

\begin{lstlisting}[style=xmlstyle, caption=TextView, label=lst:textview]
<!-- path: app/src/main/res/layout/activity_main.xml -->
<TextView
    android:id="@+id/textView"
    android:layout_width="wrap_content"
    android:layout_height="wrap_content"
    android:text="@string/app_name"
    android:textSize="24sp"
    android:textColor="@color/black"
    android:textStyle="bold" />
\end{lstlisting}

\subsection{EditText}

\begin{lstlisting}[style=xmlstyle, caption=EditText, label=lst:edittext]
<EditText
    android:id="@+id/editName"
    android:layout_width="match_parent"
    android:layout_height="wrap_content"
    android:hint="Enter hotel name"
    android:inputType="textCapWords"
    android:maxLines="1" />
\end{lstlisting}

\subsection{Button}

\begin{lstlisting}[style=xmlstyle, caption=Button, label=lst:button]
<Button
    android:id="@+id/btnSave"
    android:layout_width="wrap_content"
    android:layout_height="wrap_content"
    android:text="@string/save"
    android:backgroundTint="@color/colorPrimary" />
\end{lstlisting}

\section{Layout Types}

\subsection{LinearLayout}
\index{LinearLayout}

Arranges children in single row/column:

\begin{lstlisting}[style=xmlstyle, caption=LinearLayout vertical, label=lst:linearlayout]
<LinearLayout
    android:layout_width="match_parent"
    android:layout_height="wrap_content"
    android:orientation="vertical"
    android:padding="16dp">

    <TextView ... />
    <EditText ... />
    <Button ... />
</LinearLayout>
\end{lstlisting}

\subsection{ConstraintLayout}
\index{ConstraintLayout}

Flexible positioning with constraints:

\begin{lstlisting}[style=xmlstyle, caption=ConstraintLayout, label=lst:constraintlayout]
<androidx.constraintlayout.widget.ConstraintLayout
    android:layout_width="match_parent"
    android:layout_height="match_parent">

    <TextView
        android:id="@+id/title"
        android:layout_width="wrap_content"
        android:layout_height="wrap_content"
        app:layout_constraintTop_toTopOf="parent"
        app:layout_constraintStart_toStartOf="parent" />

    <Button
        android:layout_width="wrap_content"
        android:layout_height="wrap_content"
        app:layout_constraintTop_toBottomOf="@id/title"
        app:layout_constraintStart_toStartOf="parent" />
</androidx.constraintlayout.widget.ConstraintLayout>
\end{lstlisting}

\section{Resource Files}
\index{resources}

\subsection{strings.xml}

\begin{lstlisting}[style=xmlstyle, caption=String resources, label=lst:strings]
<!-- path: app/src/main/res/values/strings.xml -->
<resources>
    <string name="app_name">HotelApp</string>
    <string name="save">Save</string>
    <string name="cancel">Cancel</string>
    <string name="error_required">This field is required</string>
</resources>
\end{lstlisting}

Usage in Java:
\begin{lstlisting}[style=javastyle]
String appName = getString(R.string.app_name);
textView.setText(R.string.app_name);
\end{lstlisting}

\subsection{colors.xml}

\begin{lstlisting}[style=xmlstyle, caption=Color resources, label=lst:colors]
<!-- path: app/src/main/res/values/colors.xml -->
<resources>
    <color name="colorPrimary">#6200EE</color>
    <color name="colorSecondary">#03DAC6</color>
    <color name="white">#FFFFFF</color>
    <color name="black">#000000</color>
</resources>
\end{lstlisting}

\subsection{dimens.xml}

\begin{lstlisting}[style=xmlstyle, caption=Dimension resources]
<!-- path: app/src/main/res/values/dimens.xml -->
<resources>
    <dimen name="padding_small">8dp</dimen>
    <dimen name="padding_normal">16dp</dimen>
    <dimen name="padding_large">24dp</dimen>
    <dimen name="text_size_normal">16sp</dimen>
</resources>
\end{lstlisting}

\chapter{RecyclerView \& Adapters}
\index{RecyclerView}

RecyclerView displays large datasets efficiently using the ViewHolder pattern.

\section{Complete RecyclerView Implementation}

\subsection{Step 1: Add Dependency}

Already in \texttt{app/build.gradle.kts}:
\begin{lstlisting}[style=gradlestyle]
implementation("androidx.recyclerview:recyclerview:1.3.2")
\end{lstlisting}

\subsection{Step 2: Add to Layout}

\begin{lstlisting}[style=xmlstyle, caption=RecyclerView in layout, label=lst:recyclerviewxml]
<!-- path: app/src/main/res/layout/activity_main.xml -->
<androidx.recyclerview.widget.RecyclerView
    android:id="@+id/recyclerView"
    android:layout_width="match_parent"
    android:layout_height="match_parent"
    android:padding="8dp" />
\end{lstlisting}

\subsection{Step 3: Create Item Layout}

\begin{lstlisting}[style=xmlstyle, caption=Item layout, label=lst:itemlayout]
<!-- path: app/src/main/res/layout/hotel_item.xml -->
<?xml version="1.0" encoding="utf-8"?>
<androidx.cardview.widget.CardView
    xmlns:android="http://schemas.android.com/apk/res/android"
    xmlns:app="http://schemas.android.com/apk/res-auto"
    android:layout_width="match_parent"
    android:layout_height="wrap_content"
    android:layout_margin="8dp"
    app:cardCornerRadius="8dp"
    app:cardElevation="4dp">

    <LinearLayout
        android:layout_width="match_parent"
        android:layout_height="wrap_content"
        android:orientation="horizontal"
        android:padding="16dp">

        <ImageView
            android:id="@+id/hotelImage"
            android:layout_width="60dp"
            android:layout_height="60dp"
            android:src="@mipmap/ic_launcher" />

        <LinearLayout
            android:layout_width="0dp"
            android:layout_height="wrap_content"
            android:layout_weight="1"
            android:layout_marginStart="16dp"
            android:orientation="vertical">

            <TextView
                android:id="@+id/hotelName"
                android:layout_width="wrap_content"
                android:layout_height="wrap_content"
                android:textSize="18sp"
                android:textStyle="bold" />

            <TextView
                android:id="@+id/hotelLocation"
                android:layout_width="wrap_content"
                android:layout_height="wrap_content"
                android:textSize="14sp"
                android:layout_marginTop="4dp" />
        </LinearLayout>
    </LinearLayout>
</androidx.cardview.widget.CardView>
\end{lstlisting}

\subsection{Step 4: Create Adapter}

\begin{lstlisting}[style=javastyle, caption=RecyclerView Adapter, label=lst:adapter]
// path: app/src/main/java/.../adapters/HotelAdapter.java
package com.example.hotelappref.adapters;

import android.content.Context;
import android.content.Intent;
import android.view.LayoutInflater;
import android.view.ViewGroup;
import androidx.annotation.NonNull;
import androidx.recyclerview.widget.RecyclerView;
import com.example.hotelappref.HotelDetailsActivity;
import com.example.hotelappref.databinding.HotelItemBinding;
import com.example.hotelappref.models.Hotel;
import java.util.List;

public class HotelAdapter extends RecyclerView.Adapter<HotelAdapter.HotelViewHolder> {

    private final Context context;
    private final List<Hotel> hotelList;

    public HotelAdapter(Context context, List<Hotel> hotelList) {
        this.context = context;
        this.hotelList = hotelList;
    }

    @NonNull
    @Override
    public HotelViewHolder onCreateViewHolder(@NonNull ViewGroup parent, int viewType) {
        // Inflate item layout using ViewBinding
        HotelItemBinding binding = HotelItemBinding.inflate(
            LayoutInflater.from(parent.getContext()),
            parent,
            false
        );
        return new HotelViewHolder(binding);
    }

    @Override
    public void onBindViewHolder(@NonNull HotelViewHolder holder, int position) {
        Hotel hotel = hotelList.get(position);
        holder.bind(hotel);
    }

    @Override
    public int getItemCount() {
        return hotelList != null ? hotelList.size() : 0;
    }

    class HotelViewHolder extends RecyclerView.ViewHolder {
        private final HotelItemBinding binding;

        public HotelViewHolder(HotelItemBinding binding) {
            super(binding.getRoot());
            this.binding = binding;
        }

        public void bind(final Hotel hotel) {
            // Set data to views
            binding.hotelName.setText(hotel.getName());
            binding.hotelLocation.setText(hotel.getLocation());
            binding.hotelImage.setImageResource(hotel.getImageResource());

            // Handle click
            binding.getRoot().setOnClickListener(v -> {
                Intent intent = new Intent(context, HotelDetailsActivity.class);
                intent.putExtra("hotel", hotel);
                context.startActivity(intent);
            });
        }
    }
}
\end{lstlisting}

\subsection{Step 5: Setup in Activity}

\begin{lstlisting}[style=javastyle, caption=Using RecyclerView, label=lst:recyclerviewusage]
// path: app/src/main/java/.../MainActivity.java
public class MainActivity extends AppCompatActivity {

    private RecyclerView recyclerView;
    private HotelAdapter adapter;
    private List<Hotel> hotelList;

    @Override
    protected void onCreate(Bundle savedInstanceState) {
        super.onCreate(savedInstanceState);

        // Setup RecyclerView
        recyclerView = findViewById(R.id.recyclerView);
        recyclerView.setLayoutManager(new LinearLayoutManager(this));
        recyclerView.setHasFixedSize(true);

        // Prepare data
        hotelList = new ArrayList<>();
        // ... add hotels to list

        // Setup adapter
        adapter = new HotelAdapter(this, hotelList);
        recyclerView.setAdapter(adapter);
    }

    // Notify adapter of changes
    private void addHotel(Hotel hotel) {
        hotelList.add(hotel);
        adapter.notifyItemInserted(hotelList.size() - 1);
    }
}
\end{lstlisting}

\section{Updating RecyclerView}

\begin{lstlisting}[style=javastyle, caption=Adapter notifications]
// Notify all data changed
adapter.notifyDataSetChanged();

// Efficient updates
adapter.notifyItemInserted(position);
adapter.notifyItemRemoved(position);
adapter.notifyItemChanged(position);
adapter.notifyItemRangeInserted(start, count);
\end{lstlisting}

\chapter{ViewBinding}
\index{ViewBinding}

ViewBinding generates a binding class for each XML layout, providing type-safe access to views.

\section{Enable ViewBinding}

In \texttt{app/build.gradle.kts}:
\begin{lstlisting}[style=gradlestyle]
android {
    buildFeatures {
        viewBinding = true
    }
}
\end{lstlisting}

\section{Using ViewBinding in Activity}

\begin{lstlisting}[style=javastyle, caption=ViewBinding in Activity, label=lst:viewbinding]
// path: app/src/main/java/.../MainActivity.java
public class MainActivity extends AppCompatActivity {

    private ActivityMainBinding binding;

    @Override
    protected void onCreate(Bundle savedInstanceState) {
        super.onCreate(savedInstanceState);

        // Inflate binding
        binding = ActivityMainBinding.inflate(getLayoutInflater());

        // Set content view using binding root
        setContentView(binding.getRoot());

        // Access views type-safely (no findViewById!)
        binding.textView.setText("Hello");
        binding.button.setOnClickListener(v -> {
            // Handle click
        });
    }
}
\end{lstlisting}

\textbf{WHY ViewBinding?}
\begin{itemize}
    \item \textbf{Type safety}: No ClassCastException
    \item \textbf{Null safety}: Only views in layout are accessible
    \item \textbf{Better performance}: No findViewById lookups at runtime
\end{itemize}

\section{ViewBinding in Fragment}

\begin{lstlisting}[style=javastyle, caption=ViewBinding in Fragment]
public class MyFragment extends Fragment {

    private FragmentMyBinding binding;

    @Override
    public View onCreateView(LayoutInflater inflater, ViewGroup container,
                             Bundle savedInstanceState) {
        binding = FragmentMyBinding.inflate(inflater, container, false);
        return binding.getRoot();
    }

    @Override
    public void onDestroyView() {
        super.onDestroyView();
        binding = null;  // Prevent memory leaks
    }
}
\end{lstlisting}

\chapter{Menus, Toasts \& Snackbars}
\index{Menu}\index{Toast}\index{Snackbar}

\section{Options Menu}

\subsection{Create Menu Resource}

\begin{lstlisting}[style=xmlstyle, caption=Menu XML, label=lst:menu]
<!-- path: app/src/main/res/menu/main_menu.xml -->
<?xml version="1.0" encoding="utf-8"?>
<menu xmlns:android="http://schemas.android.com/apk/res/android"
    xmlns:app="http://schemas.android.com/apk/res-auto">

    <item
        android:id="@+id/action_search"
        android:icon="@android:drawable/ic_menu_search"
        android:title="Search"
        app:showAsAction="ifRoom" />

    <item
        android:id="@+id/action_settings"
        android:title="Settings"
        app:showAsAction="never" />

    <item
        android:id="@+id/action_about"
        android:title="About"
        app:showAsAction="never" />
</menu>
\end{lstlisting}

\subsection{Inflate and Handle Menu}

\begin{lstlisting}[style=javastyle, caption=Menu in Activity, label=lst:menuactivity]
// path: app/src/main/java/.../MainActivity.java
@Override
public boolean onCreateOptionsMenu(Menu menu) {
    getMenuInflater().inflate(R.menu.main_menu, menu);
    return true;
}

@Override
public boolean onOptionsItemSelected(MenuItem item) {
    int id = item.getItemId();

    if (id == R.id.action_search) {
        // Handle search
        Toast.makeText(this, "Search clicked", Toast.LENGTH_SHORT).show();
        return true;
    } else if (id == R.id.action_settings) {
        // Open settings
        startActivity(new Intent(this, SettingsActivity.class));
        return true;
    } else if (id == R.id.action_about) {
        // Show about dialog
        showAboutDialog();
        return true;
    }

    return super.onOptionsItemSelected(item);
}
\end{lstlisting}

\section{Toast Messages}
\index{Toast}

Quick, non-intrusive messages:

\begin{lstlisting}[style=javastyle, caption=Toast examples, label=lst:toast]
// Short toast (2 seconds)
Toast.makeText(this, "Item saved", Toast.LENGTH_SHORT).show();

// Long toast (3.5 seconds)
Toast.makeText(this, "Item deleted", Toast.LENGTH_LONG).show();

// Custom position
Toast toast = Toast.makeText(this, "Message", Toast.LENGTH_SHORT);
toast.setGravity(Gravity.TOP | Gravity.CENTER, 0, 0);
toast.show();
\end{lstlisting}

\section{Snackbar}
\index{Snackbar}

Material Design message with optional action:

\begin{lstlisting}[style=javastyle, caption=Snackbar examples, label=lst:snackbar]
// Simple snackbar
Snackbar.make(view, "Item deleted", Snackbar.LENGTH_SHORT).show();

// Snackbar with action
Snackbar.make(view, "Item deleted", Snackbar.LENGTH_LONG)
    .setAction("UNDO", v -> {
        // Restore deleted item
    })
    .show();

// Custom colors
Snackbar snackbar = Snackbar.make(view, "Message", Snackbar.LENGTH_SHORT);
snackbar.setBackgroundTint(Color.RED);
snackbar.setTextColor(Color.WHITE);
snackbar.show();
\end{lstlisting}

\chapter{Dialogs}
\index{Dialog}

\section{AlertDialog}

\begin{lstlisting}[style=javastyle, caption=AlertDialog examples, label=lst:alertdialog]
// Simple alert
new AlertDialog.Builder(this)
    .setTitle("Delete Hotel")
    .setMessage("Are you sure you want to delete this hotel?")
    .setPositiveButton("Delete", (dialog, which) -> {
        // Delete hotel
    })
    .setNegativeButton("Cancel", null)
    .show();

// List dialog
String[] items = {"Option 1", "Option 2", "Option 3"};
new AlertDialog.Builder(this)
    .setTitle("Choose an option")
    .setItems(items, (dialog, which) -> {
        Toast.makeText(this, "Selected: " + items[which], Toast.LENGTH_SHORT).show();
    })
    .show();

// Single choice dialog
new AlertDialog.Builder(this)
    .setTitle("Select category")
    .setSingleChoiceItems(items, 0, (dialog, which) -> {
        // Handle selection
    })
    .setPositiveButton("OK", (dialog, which) -> {
        // Confirm selection
    })
    .show();

// Multi-choice dialog
boolean[] checkedItems = {false, false, false};
new AlertDialog.Builder(this)
    .setTitle("Select multiple")
    .setMultiChoiceItems(items, checkedItems, (dialog, which, isChecked) -> {
        checkedItems[which] = isChecked;
    })
    .setPositiveButton("OK", null)
    .show();
\end{lstlisting}

\chapter{Form Validation}
\index{validation}

\section{EditText Validation}

\begin{lstlisting}[style=javastyle, caption=Form validation, label=lst:validation]
// path: app/src/main/java/.../AddHotelActivity.java
private void saveHotel() {
    // Get input
    String name = binding.editName.getText().toString().trim();
    String phone = binding.editPhone.getText().toString().trim();

    // Validate name
    if (TextUtils.isEmpty(name)) {
        binding.editName.setError("Name is required");
        binding.editName.requestFocus();
        return;
    }

    // Validate phone
    if (TextUtils.isEmpty(phone)) {
        binding.editPhone.setError("Phone is required");
        binding.editPhone.requestFocus();
        return;
    }

    // Validate email format
    if (!android.util.Patterns.EMAIL_ADDRESS.matcher(email).matches()) {
        binding.editEmail.setError("Invalid email");
        binding.editEmail.requestFocus();
        return;
    }

    // Validate phone format
    if (!android.util.Patterns.PHONE.matcher(phone).matches()) {
        binding.editPhone.setError("Invalid phone number");
        binding.editPhone.requestFocus();
        return;
    }

    // All valid - proceed
    saveToDatabase();
}
\end{lstlisting}

\section{Common Validation Patterns}

\begin{lstlisting}[style=javastyle, caption=Validation utilities]
// Check if empty
if (TextUtils.isEmpty(text)) { }

// Minimum length
if (text.length() < 6) {
    editText.setError("Minimum 6 characters");
}

// Email validation
if (!Patterns.EMAIL_ADDRESS.matcher(email).matches()) { }

// Phone validation
if (!Patterns.PHONE.matcher(phone).matches()) { }

// URL validation
if (!Patterns.WEB_URL.matcher(url).matches()) { }

// Numeric validation
try {
    int value = Integer.parseInt(text);
} catch (NumberFormatException e) {
    editText.setError("Must be a number");
}

// Match two fields (e.g., confirm password)
if (!password.equals(confirmPassword)) {
    binding.editConfirmPassword.setError("Passwords don't match");
}
\end{lstlisting}

\chapter{Permissions}
\index{permissions}

\section{Declaring Permissions}

In \texttt{AndroidManifest.xml}:

\begin{lstlisting}[style=xmlstyle, caption=Permission declaration, label=lst:permissions]
<!-- path: app/src/main/AndroidManifest.xml -->
<manifest ...>
    <!-- Internet permission (no runtime request needed) -->
    <uses-permission android:name="android.permission.INTERNET" />

    <!-- Dangerous permissions (require runtime request) -->
    <uses-permission android:name="android.permission.CALL_PHONE" />
    <uses-permission android:name="android.permission.CAMERA" />
    <uses-permission android:name="android.permission.READ_EXTERNAL_STORAGE" />

    <application ...>
    </application>
</manifest>
\end{lstlisting}

\section{Runtime Permissions}

For dangerous permissions (API 23+):

\begin{lstlisting}[style=javastyle, caption=Runtime permission, label=lst:runtimepermission]
// Check if permission granted
if (ContextCompat.checkSelfPermission(this,
        Manifest.permission.CALL_PHONE) == PackageManager.PERMISSION_GRANTED) {
    // Permission already granted
    makePhoneCall();
} else {
    // Request permission
    ActivityCompat.requestPermissions(this,
        new String[]{Manifest.permission.CALL_PHONE},
        REQUEST_CALL_PHONE);
}

// Handle permission result
@Override
public void onRequestPermissionsResult(int requestCode,
        String[] permissions, int[] grantResults) {
    super.onRequestPermissionsResult(requestCode, permissions, grantResults);

    if (requestCode == REQUEST_CALL_PHONE) {
        if (grantResults.length > 0 &&
                grantResults[0] == PackageManager.PERMISSION_GRANTED) {
            // Permission granted
            makePhoneCall();
        } else {
            // Permission denied
            Toast.makeText(this, "Permission denied", Toast.LENGTH_SHORT).show();
        }
    }
}
\end{lstlisting}

\chapter{Storage: Room Database}
\index{Room}

Room is Android's recommended database library, providing an abstraction over SQLite.

\section{Setup Room}

\subsection{Add Dependencies}

\begin{lstlisting}[style=gradlestyle, caption=Room dependencies]
// path: app/build.gradle.kts
dependencies {
    implementation("androidx.room:room-runtime:2.6.1")
    annotationProcessor("androidx.room:room-compiler:2.6.1")
}
\end{lstlisting}

\section{Room Components}

\subsection{1. Entity (Model)}

\begin{lstlisting}[style=javastyle, caption=Room Entity, label=lst:entity]
// path: app/src/main/java/.../models/Hotel.java
package com.example.hotelappref.models;

import androidx.room.Entity;
import androidx.room.PrimaryKey;
import java.io.Serializable;

@Entity(tableName = "hotels")
public class Hotel implements Serializable {

    @PrimaryKey(autoGenerate = true)
    private long id;

    private String name;
    private String phone;
    private String website;
    private String location;
    private String nearby;
    private String food;
    private int imageResource;

    public Hotel(String name, String phone, String website,
                 String location, String nearby, String food, int imageResource) {
        this.name = name;
        this.phone = phone;
        this.website = website;
        this.location = location;
        this.nearby = nearby;
        this.food = food;
        this.imageResource = imageResource;
    }

    // Getters and setters
    public long getId() { return id; }
    public void setId(long id) { this.id = id; }

    public String getName() { return name; }
    public void setName(String name) { this.name = name; }

    // ... other getters/setters
}
\end{lstlisting}

\subsection{2. DAO (Data Access Object)}

\begin{lstlisting}[style=javastyle, caption=Room DAO, label=lst:dao]
// path: app/src/main/java/.../database/HotelDao.java
package com.example.hotelappref.database;

import androidx.room.Dao;
import androidx.room.Delete;
import androidx.room.Insert;
import androidx.room.Query;
import androidx.room.Update;
import com.example.hotelappref.models.Hotel;
import java.util.List;

@Dao
public interface HotelDao {

    @Insert
    long insert(Hotel hotel);

    @Update
    void update(Hotel hotel);

    @Delete
    void delete(Hotel hotel);

    @Query("SELECT * FROM hotels ORDER BY name ASC")
    List<Hotel> getAllHotels();

    @Query("SELECT * FROM hotels WHERE id = :hotelId")
    Hotel getHotelById(long hotelId);

    @Query("SELECT * FROM hotels WHERE name LIKE '%' || :searchQuery || '%' ORDER BY name ASC")
    List<Hotel> searchHotelsByName(String searchQuery);

    @Query("DELETE FROM hotels")
    void deleteAllHotels();

    @Query("SELECT COUNT(*) FROM hotels")
    int getHotelCount();
}
\end{lstlisting}

\subsection{3. Database Class}

\begin{lstlisting}[style=javastyle, caption=Room Database, label=lst:database]
// path: app/src/main/java/.../database/AppDatabase.java
package com.example.hotelappref.database;

import android.content.Context;
import androidx.room.Database;
import androidx.room.Room;
import androidx.room.RoomDatabase;
import com.example.hotelappref.models.Hotel;

@Database(entities = {Hotel.class}, version = 1, exportSchema = false)
public abstract class AppDatabase extends RoomDatabase {

    public abstract HotelDao hotelDao();

    private static volatile AppDatabase INSTANCE;

    public static AppDatabase getInstance(Context context) {
        if (INSTANCE == null) {
            synchronized (AppDatabase.class) {
                if (INSTANCE == null) {
                    INSTANCE = Room.databaseBuilder(
                        context.getApplicationContext(),
                        AppDatabase.class,
                        "hotel_database"
                    )
                    .allowMainThreadQueries()  // For simplicity; use background thread in production
                    .build();
                }
            }
        }
        return INSTANCE;
    }
}
\end{lstlisting}

\section{Using Room in Activity}

\begin{lstlisting}[style=javastyle, caption=Using Room, label=lst:roomusage]
// path: app/src/main/java/.../MainActivity.java
public class MainActivity extends AppCompatActivity {

    private AppDatabase database;
    private HotelDao hotelDao;
    private List<Hotel> hotelList;

    @Override
    protected void onCreate(Bundle savedInstanceState) {
        super.onCreate(savedInstanceState);

        // Get database instance
        database = AppDatabase.getInstance(this);
        hotelDao = database.hotelDao();

        // Load data
        loadHotelsFromDatabase();
    }

    private void loadHotelsFromDatabase() {
        hotelList = new ArrayList<>(hotelDao.getAllHotels());
    }

    private void addHotel() {
        Hotel hotel = new Hotel("Grand Plaza", "+9611234567",
            "www.grandplaza.com", "Beirut", "Downtown",
            "Lebanese Cuisine", R.mipmap.ic_launcher);

        long id = hotelDao.insert(hotel);
        Toast.makeText(this, "Hotel added with ID: " + id, Toast.LENGTH_SHORT).show();
    }

    private void updateHotel(Hotel hotel) {
        hotel.setName("Updated Name");
        hotelDao.update(hotel);
    }

    private void deleteHotel(Hotel hotel) {
        hotelDao.delete(hotel);
    }

    private void searchHotels(String query) {
        List<Hotel> results = hotelDao.searchHotelsByName(query);
        // Update UI with results
    }
}
\end{lstlisting}

\section{SharedPreferences}
\index{SharedPreferences}

For simple key-value storage:

\begin{lstlisting}[style=javastyle, caption=SharedPreferences, label=lst:sharedprefs]
// Save data
SharedPreferences prefs = getSharedPreferences("MyPrefs", MODE_PRIVATE);
SharedPreferences.Editor editor = prefs.edit();
editor.putString("username", "john_doe");
editor.putInt("user_id", 123);
editor.putBoolean("is_logged_in", true);
editor.apply();  // async save

// Read data
String username = prefs.getString("username", "");  // default ""
int userId = prefs.getInt("user_id", -1);  // default -1
boolean isLoggedIn = prefs.getBoolean("is_logged_in", false);

// Remove key
editor.remove("username");
editor.apply();

// Clear all
editor.clear();
editor.apply();
\end{lstlisting}

\chapter{Themes, Colors \& Styles}
\index{themes}\index{styles}

\section{themes.xml}

\begin{lstlisting}[style=xmlstyle, caption=Theme definition, label=lst:themes]
<!-- path: app/src/main/res/values/themes.xml -->
<resources>
    <style name="Base.Theme.HotelAppRef" parent="Theme.Material3.Light.NoActionBar">
        <item name="colorPrimary">@color/colorPrimary</item>
        <item name="colorSecondary">@color/colorSecondary</item>
        <item name="android:statusBarColor">@color/colorPrimary</item>
    </style>

    <style name="Theme.HotelAppRef" parent="Base.Theme.HotelAppRef" />
</resources>
\end{lstlisting}

\section{Custom Styles}

\begin{lstlisting}[style=xmlstyle, caption=Custom styles]
<!-- path: app/src/main/res/values/styles.xml -->
<resources>
    <!-- Button style -->
    <style name="PrimaryButton" parent="Widget.MaterialComponents.Button">
        <item name="android:textColor">@color/white</item>
        <item name="backgroundTint">@color/colorPrimary</item>
        <item name="android:padding">16dp</item>
    </style>

    <!-- Title text style -->
    <style name="TitleText">
        <item name="android:textSize">24sp</item>
        <item name="android:textStyle">bold</item>
        <item name="android:textColor">@color/black</item>
    </style>

    <!-- EditText style -->
    <style name="FormEditText">
        <item name="android:layout_width">match_parent</item>
        <item name="android:layout_height">wrap_content</item>
        <item name="android:padding">12dp</item>
        <item name="android:background">@drawable/edittext_background</item>
    </style>
</resources>

<!-- Usage in XML -->
<Button
    style="@style/PrimaryButton"
    android:text="Save" />

<TextView
    style="@style/TitleText"
    android:text="Hotels" />
\end{lstlisting}

\chapter{AndroidManifest.xml Essentials}
\index{Manifest}

\section{Complete Manifest Example}

\begin{lstlisting}[style=xmlstyle, caption=AndroidManifest.xml, label=lst:manifest]
<!-- path: app/src/main/AndroidManifest.xml -->
<?xml version="1.0" encoding="utf-8"?>
<manifest xmlns:android="http://schemas.android.com/apk/res/android"
    xmlns:tools="http://schemas.android.com/tools">

    <!-- Permissions -->
    <uses-permission android:name="android.permission.INTERNET" />
    <uses-permission android:name="android.permission.CALL_PHONE" />

    <application
        android:allowBackup="true"
        android:icon="@mipmap/ic_launcher"
        android:label="@string/app_name"
        android:roundIcon="@mipmap/ic_launcher_round"
        android:supportsRtl="true"
        android:theme="@style/Base.Theme.HotelAppRef"
        tools:targetApi="31">

        <!-- Main Activity (Launcher) -->
        <activity
            android:name=".MainActivity"
            android:exported="true">
            <intent-filter>
                <action android:name="android.intent.action.MAIN" />
                <category android:name="android.intent.category.LAUNCHER" />
            </intent-filter>
        </activity>

        <!-- Other Activities -->
        <activity
            android:name=".AddHotelActivity"
            android:exported="false"
            android:parentActivityName=".MainActivity"
            android:windowSoftInputMode="adjustResize">
            <meta-data
                android:name="android.support.PARENT_ACTIVITY"
                android:value=".MainActivity" />
        </activity>

        <activity
            android:name=".HotelDetailsActivity"
            android:exported="false"
            android:parentActivityName=".MainActivity" />

    </application>

</manifest>
\end{lstlisting}

\section{Key Manifest Attributes}

\begin{itemize}
    \item \texttt{android:name}: Activity class name (dot prefix for same package)
    \item \texttt{android:exported}: Whether other apps can start this activity
    \item \texttt{android:label}: Activity title (shown in action bar)
    \item \texttt{android:theme}: Activity-specific theme
    \item \texttt{android:parentActivityName}: Enables up navigation
    \item \texttt{android:windowSoftInputMode}: How keyboard affects layout
    \item \texttt{android:screenOrientation}: Lock orientation (portrait/landscape)
    \item \texttt{android:configChanges}: Handle config changes manually
\end{itemize}

\part{Midterm Hotel App Walkthrough}

\chapter{Hotel App Requirements}

The Hotel App manages a list of hotels in Lebanon. Users can:

\begin{enumerate}
    \item View list of hotels (RecyclerView)
    \item Add new hotels (form with validation)
    \item View hotel details
    \item Persist data (Room database)
\end{enumerate}

\section{Data Model}

Hotel entity with fields:
\begin{itemize}
    \item \textbf{id}: Unique identifier (auto-generated)
    \item \textbf{name}: Hotel name (required)
    \item \textbf{phone}: Contact number (required)
    \item \textbf{website}: Hotel website
    \item \textbf{location}: Physical address (required)
    \item \textbf{nearby}: Nearby attractions
    \item \textbf{food}: Dining options
    \item \textbf{imageResource}: Icon resource ID
\end{itemize}

\chapter{MainActivity Implementation}

\section{Complete Code}

\begin{lstlisting}[style=javastyle, caption=MainActivity.java, label=lst:mainactivity]
// path: app/src/main/java/com/example/hotelappref/MainActivity.java
package com.example.hotelappref;

import android.content.Intent;
import android.os.Bundle;
import androidx.activity.result.ActivityResultLauncher;
import androidx.activity.result.contract.ActivityResultContracts;
import androidx.appcompat.app.AppCompatActivity;
import androidx.recyclerview.widget.LinearLayoutManager;
import com.example.hotelappref.adapters.HotelAdapter;
import com.example.hotelappref.database.AppDatabase;
import com.example.hotelappref.database.HotelDao;
import com.example.hotelappref.databinding.ActivityMainBinding;
import com.example.hotelappref.models.Hotel;
import java.util.ArrayList;
import java.util.List;

/**
 * MainActivity - The main screen displaying the list of hotels.
 *
 * KEY CONCEPTS:
 * - ViewBinding: Type-safe way to access views (no findViewById!)
 * - RecyclerView: Efficient list display with ViewHolder pattern
 * - Room Database: Persistent storage for hotel data
 * - ActivityResultLauncher: Modern way to handle activity results
 */
public class MainActivity extends AppCompatActivity {

    private ActivityMainBinding binding;
    private HotelAdapter adapter;
    private List<Hotel> hotelList;
    private AppDatabase database;
    private HotelDao hotelDao;
    private ActivityResultLauncher<Intent> addHotelLauncher;

    @Override
    protected void onCreate(Bundle savedInstanceState) {
        super.onCreate(savedInstanceState);

        // ViewBinding setup
        binding = ActivityMainBinding.inflate(getLayoutInflater());
        setContentView(binding.getRoot());

        // Action bar
        if (getSupportActionBar() != null) {
            getSupportActionBar().setTitle("Hotels in Lebanon");
        }

        // Initialize database
        database = AppDatabase.getInstance(this);
        hotelDao = database.hotelDao();
        AppDatabase.populateInitialData(this);

        // Initialize Activity Result Launcher
        initializeAddHotelLauncher();

        // Setup RecyclerView
        binding.recyclerView.setLayoutManager(new LinearLayoutManager(this));
        binding.recyclerView.setHasFixedSize(true);

        // Load data
        loadHotelsFromDatabase();

        // Setup adapter
        adapter = new HotelAdapter(this, hotelList);
        binding.recyclerView.setAdapter(adapter);

        // FAB click listener
        binding.fabAddHotel.setOnClickListener(v -> {
            Intent intent = new Intent(MainActivity.this, AddHotelActivity.class);
            addHotelLauncher.launch(intent);
        });
    }

    private void initializeAddHotelLauncher() {
        addHotelLauncher = registerForActivityResult(
            new ActivityResultContracts.StartActivityForResult(),
            result -> {
                if (result.getResultCode() == RESULT_OK) {
                    loadHotelsFromDatabase();
                    adapter.notifyDataSetChanged();
                    if (hotelList.size() > 0) {
                        binding.recyclerView.smoothScrollToPosition(hotelList.size() - 1);
                    }
                }
            }
        );
    }

    private void loadHotelsFromDatabase() {
        hotelList = new ArrayList<>(hotelDao.getAllHotels());
    }
}
\end{lstlisting}

\section{Why It Works}

\begin{enumerate}
    \item \textbf{ViewBinding} eliminates findViewById and provides type safety
    \item \textbf{Room Database} persists data across app restarts
    \item \textbf{RecyclerView} efficiently displays large lists
    \item \textbf{ActivityResultLauncher} replaces deprecated startActivityForResult
    \item \textbf{Singleton pattern} ensures one database instance
\end{enumerate}

\chapter{AddHotelActivity Implementation}

\begin{lstlisting}[style=javastyle, caption=AddHotelActivity.java, label=lst:addhotelactivity]
// path: app/src/main/java/com/example/hotelappref/AddHotelActivity.java
package com.example.hotelappref;

import android.os.Bundle;
import android.text.TextUtils;
import android.widget.Toast;
import androidx.appcompat.app.AppCompatActivity;
import com.example.hotelappref.database.AppDatabase;
import com.example.hotelappref.database.HotelDao;
import com.example.hotelappref.databinding.ActivityAddHotelBinding;
import com.example.hotelappref.models.Hotel;

public class AddHotelActivity extends AppCompatActivity {

    private ActivityAddHotelBinding binding;
    private HotelDao hotelDao;

    @Override
    protected void onCreate(Bundle savedInstanceState) {
        super.onCreate(savedInstanceState);
        binding = ActivityAddHotelBinding.inflate(getLayoutInflater());
        setContentView(binding.getRoot());

        // Initialize database
        AppDatabase database = AppDatabase.getInstance(this);
        hotelDao = database.hotelDao();

        // Setup action bar
        if (getSupportActionBar() != null) {
            getSupportActionBar().setDisplayHomeAsUpEnabled(true);
            getSupportActionBar().setTitle("Add New Hotel");
        }

        // Button listeners
        binding.btnSave.setOnClickListener(v -> saveHotel());
        binding.btnCancel.setOnClickListener(v -> {
            setResult(RESULT_CANCELED);
            finish();
        });
    }

    private void saveHotel() {
        // Get inputs
        String name = binding.editName.getText().toString().trim();
        String phone = binding.editPhone.getText().toString().trim();
        String website = binding.editWebsite.getText().toString().trim();
        String location = binding.editLocation.getText().toString().trim();
        String nearby = binding.editNearby.getText().toString().trim();
        String food = binding.editFood.getText().toString().trim();

        // Validation
        if (TextUtils.isEmpty(name)) {
            binding.editName.setError("Name is required");
            binding.editName.requestFocus();
            return;
        }

        if (TextUtils.isEmpty(phone)) {
            binding.editPhone.setError("Phone is required");
            binding.editPhone.requestFocus();
            return;
        }

        if (TextUtils.isEmpty(location)) {
            binding.editLocation.setError("Location is required");
            binding.editLocation.requestFocus();
            return;
        }

        // Defaults for optional fields
        if (TextUtils.isEmpty(website)) website = "www.hotel.com";
        if (TextUtils.isEmpty(nearby)) nearby = "No nearby attractions listed";
        if (TextUtils.isEmpty(food)) food = "Restaurant available";

        // Create and save hotel
        Hotel newHotel = new Hotel(name, phone, website, location, nearby, food,
            R.mipmap.ic_launcher);
        hotelDao.insert(newHotel);

        // Return success
        setResult(RESULT_OK);
        Toast.makeText(this, "Hotel added successfully!", Toast.LENGTH_SHORT).show();
        finish();
    }

    @Override
    public boolean onSupportNavigateUp() {
        onBackPressed();
        return true;
    }
}
\end{lstlisting}

\chapter{Common Exam Tweaks}

\section{Add a New Field}

\textbf{Example:} Add a "rating" field to Hotel

\begin{enumerate}
    \item Update Hotel model:
\begin{lstlisting}[style=javastyle]
private float rating;

public float getRating() { return rating; }
public void setRating(float rating) { this.rating = rating; }
\end{lstlisting}

    \item Update database version in AppDatabase:
\begin{lstlisting}[style=javastyle]
@Database(entities = {Hotel.class}, version = 2)  // Increment version
\end{lstlisting}

    \item Add to layout XML:
\begin{lstlisting}[style=xmlstyle]
<RatingBar
    android:id="@+id/ratingBar"
    android:layout_width="wrap_content"
    android:layout_height="wrap_content" />
\end{lstlisting}

    \item Update AddHotelActivity:
\begin{lstlisting}[style=javastyle]
float rating = binding.ratingBar.getRating();
hotel.setRating(rating);
\end{lstlisting}
\end{enumerate}

\section{Add Search Functionality}

\begin{enumerate}
    \item Add SearchView to menu:
\begin{lstlisting}[style=xmlstyle]
<item
    android:id="@+id/action_search"
    android:title="Search"
    app:showAsAction="ifRoom|collapseActionView"
    app:actionViewClass="androidx.appcompat.widget.SearchView" />
\end{lstlisting}

    \item Handle in MainActivity:
\begin{lstlisting}[style=javastyle]
@Override
public boolean onCreateOptionsMenu(Menu menu) {
    getMenuInflater().inflate(R.menu.main_menu, menu);

    MenuItem searchItem = menu.findItem(R.id.action_search);
    SearchView searchView = (SearchView) searchItem.getActionView();

    searchView.setOnQueryTextListener(new SearchView.OnQueryTextListener() {
        @Override
        public boolean onQueryTextSubmit(String query) {
            searchHotels(query);
            return true;
        }

        @Override
        public boolean onQueryTextChange(String newText) {
            searchHotels(newText);
            return true;
        }
    });

    return true;
}

private void searchHotels(String query) {
    if (TextUtils.isEmpty(query)) {
        hotelList = new ArrayList<>(hotelDao.getAllHotels());
    } else {
        hotelList = new ArrayList<>(hotelDao.searchHotelsByName(query));
    }
    adapter = new HotelAdapter(this, hotelList);
    binding.recyclerView.setAdapter(adapter);
}
\end{lstlisting}
\end{enumerate}

\section{Add Delete Functionality}

\begin{enumerate}
    \item Add delete method in HotelAdapter:
\begin{lstlisting}[style=javastyle]
public void bind(final Hotel hotel) {
    // ... existing binding code

    binding.getRoot().setOnLongClickListener(v -> {
        new AlertDialog.Builder(context)
            .setTitle("Delete Hotel")
            .setMessage("Delete " + hotel.getName() + "?")
            .setPositiveButton("Delete", (dialog, which) -> {
                hotelDao.delete(hotel);
                hotelList.remove(getAdapterPosition());
                notifyItemRemoved(getAdapterPosition());
                Toast.makeText(context, "Deleted", Toast.LENGTH_SHORT).show();
            })
            .setNegativeButton("Cancel", null)
            .show();
        return true;
    });
}
\end{lstlisting}
\end{enumerate}

\part{Labs Cheatsheet}

\chapter{FriendsList Lab}

\section{Requirements}

Build an app to manage favorite books with:
\begin{itemize}
    \item ListView showing books (title + genre)
    \item FAB to add books via explicit intent
    \item Form with EditText (title) and Spinner (genre)
    \item Form validation (non-empty fields)
    \item Click book to share via WhatsApp (implicit intent)
\end{itemize}

\section{Key Code Snippets}

\subsection{Spinner Setup}

\begin{lstlisting}[style=javastyle, caption=Spinner in Activity]
// In layout XML
<Spinner
    android:id="@+id/spinnerGenre"
    android:layout_width="match_parent"
    android:layout_height="wrap_content" />

// In Activity
Spinner spinner = findViewById(R.id.spinnerGenre);
ArrayAdapter<CharSequence> adapter = ArrayAdapter.createFromResource(
    this,
    R.array.genres_array,  // Define in strings.xml
    android.R.layout.simple_spinner_item
);
adapter.setDropDownViewResource(android.R.layout.simple_spinner_dropdown_item);
spinner.setAdapter(adapter);

// Get selected value
String genre = spinner.getSelectedItem().toString();
\end{lstlisting}

\subsection{Implicit Intent for WhatsApp}

\begin{lstlisting}[style=javastyle, caption=Share to WhatsApp]
String phoneNumber = "+96176867167";
String message = "Check out this book: " + bookTitle + " - " + genre;

Intent intent = new Intent(Intent.ACTION_VIEW);
intent.setData(Uri.parse("https://api.whatsapp.com/send?phone=" + phoneNumber +
    "&text=" + Uri.encode(message)));
startActivity(intent);
\end{lstlisting}

\subsection{ListView Setup}

\begin{lstlisting}[style=javastyle, caption=ListView with ArrayAdapter]
ListView listView = findViewById(R.id.listView);
List<String> books = new ArrayList<>();
books.add("Book Title - Genre");

ArrayAdapter<String> adapter = new ArrayAdapter<>(
    this,
    android.R.layout.simple_list_item_1,
    books
);
listView.setAdapter(adapter);

// Handle clicks
listView.setOnItemClickListener((parent, view, position, id) -> {
    String selectedBook = books.get(position);
    // Share to WhatsApp
});
\end{lstlisting}

\part{Debugging \& Gotchas}

\chapter{Common Gradle Errors}

\section{Gradle Sync Failed}

\textbf{Solution:}
\begin{enumerate}
    \item File → Invalidate Caches → Restart
    \item Delete \texttt{.gradle} and \texttt{.idea} folders
    \item File → Sync Project with Gradle Files
\end{enumerate}

\section{Failed to Resolve Dependency}

\textbf{Solution:}
\begin{itemize}
    \item Check internet connection
    \item Verify dependency version exists
    \item Add/verify repositories in \texttt{settings.gradle.kts}:
\begin{lstlisting}[style=gradlestyle]
repositories {
    google()
    mavenCentral()
}
\end{lstlisting}
\end{itemize}

\chapter{Common Manifest Errors}

\section{Activity Not Registered}

\textbf{Error:} Unable to find explicit activity class

\textbf{Solution:} Add activity to AndroidManifest.xml:
\begin{lstlisting}[style=xmlstyle]
<activity android:name=".YourActivity" />
\end{lstlisting}

\section{Permission Denied}

\textbf{Error:} java.lang.SecurityException: Permission denied

\textbf{Solution:}
\begin{enumerate}
    \item Add permission to Manifest
    \item Request runtime permission (dangerous permissions)
\end{enumerate}

\chapter{Common Resource Errors}

\section{Resource Not Found}

\textbf{Error:} android.content.res.Resources\$NotFoundException

\textbf{Causes:}
\begin{itemize}
    \item Typo in resource ID (\texttt{R.id.textview} vs \texttt{R.id.textView})
    \item Resource in wrong folder
    \item Invalid XML
\end{itemize}

\textbf{Solution:}
\begin{enumerate}
    \item Build → Clean Project
    \item Build → Rebuild Project
    \item Verify resource name matches exactly
\end{enumerate}

\section{InflateException}

\textbf{Error:} android.view.InflateException: Binary XML file line X

\textbf{Causes:}
\begin{itemize}
    \item Invalid XML syntax
    \item Missing required attribute
    \item Wrong parent class in custom view
\end{itemize}

\textbf{Solution:}
\begin{itemize}
    \item Check Logcat for detailed error
    \item Validate XML (look for red underlines)
    \item Check line number mentioned in error
\end{itemize}

\chapter{Common Runtime Errors}

\section{NullPointerException}

\textbf{Most common causes:}
\begin{enumerate}
    \item View not initialized:
\begin{lstlisting}[style=javastyle]
// Wrong: textView not initialized
textView.setText("text");  // NPE!

// Right:
TextView textView = findViewById(R.id.textView);
textView.setText("text");
\end{lstlisting}

    \item Intent extra not provided:
\begin{lstlisting}[style=javastyle]
// Check before using
String name = getIntent().getStringExtra("name");
if (name != null) {
    // Use name
}
\end{lstlisting}

    \item Database not initialized:
\begin{lstlisting}[style=javastyle]
// Initialize first
database = AppDatabase.getInstance(this);
hotelDao = database.hotelDao();
\end{lstlisting}
\end{enumerate}

\section{App Crashes on Rotation}

\textbf{Cause:} Activity recreated, state lost

\textbf{Solution:} Save and restore state:
\begin{lstlisting}[style=javastyle]
@Override
protected void onSaveInstanceState(Bundle outState) {
    super.onSaveInstanceState(outState);
    outState.putString("data", myData);
}

@Override
protected void onCreate(Bundle savedInstanceState) {
    super.onCreate(savedInstanceState);
    if (savedInstanceState != null) {
        myData = savedInstanceState.getString("data");
    }
}
\end{lstlisting}

\chapter{Debugging Tips}

\section{Using Logcat Effectively}

\begin{lstlisting}[style=javastyle]
private static final String TAG = "MainActivity";

// Debug info
Log.d(TAG, "onCreate: hotelList size = " + hotelList.size());

// Errors
try {
    // risky operation
} catch (Exception e) {
    Log.e(TAG, "Error: " + e.getMessage(), e);
}
\end{lstlisting}

\section{Toast for Quick Debugging}

\begin{lstlisting}[style=javastyle]
Toast.makeText(this, "Variable value: " + myVar, Toast.LENGTH_SHORT).show();
\end{lstlisting}

\section{Breakpoints}

\begin{enumerate}
    \item Click left gutter to set breakpoint
    \item Debug button (not Run)
    \item Inspect variables when paused
    \item Step through code (F8, F7)
\end{enumerate}

\appendix

\chapter{File Placement Map}

\begin{longtable}{|p{6cm}|p{9cm}|}
\hline
\textbf{File Type} & \textbf{Location} \\
\hline
Activity Java class & \texttt{app/src/main/java/com/example/hotelappref/} \\
\hline
Model class & \texttt{app/src/main/java/.../models/} \\
\hline
Adapter class & \texttt{app/src/main/java/.../adapters/} \\
\hline
DAO interface & \texttt{app/src/main/java/.../database/} \\
\hline
Database class & \texttt{app/src/main/java/.../database/} \\
\hline
Layout XML & \texttt{app/src/main/res/layout/} \\
\hline
String resources & \texttt{app/src/main/res/values/strings.xml} \\
\hline
Color resources & \texttt{app/src/main/res/values/colors.xml} \\
\hline
Theme/Style & \texttt{app/src/main/res/values/themes.xml} \\
\hline
Dimensions & \texttt{app/src/main/res/values/dimens.xml} \\
\hline
Menu XML & \texttt{app/src/main/res/menu/} \\
\hline
App icons (mdpi) & \texttt{app/src/main/res/mipmap-mdpi/} \\
\hline
App icons (hdpi) & \texttt{app/src/main/res/mipmap-hdpi/} \\
\hline
App icons (xhdpi) & \texttt{app/src/main/res/mipmap-xhdpi/} \\
\hline
App icons (xxhdpi) & \texttt{app/src/main/res/mipmap-xxhdpi/} \\
\hline
App icons (xxxhdpi) & \texttt{app/src/main/res/mipmap-xxxhdpi/} \\
\hline
Drawable images & \texttt{app/src/main/res/drawable/} \\
\hline
Manifest & \texttt{app/src/main/AndroidManifest.xml} \\
\hline
App Gradle & \texttt{app/build.gradle.kts} \\
\hline
Project Gradle & \texttt{build.gradle.kts} \\
\hline
Settings Gradle & \texttt{settings.gradle.kts} \\
\hline
Gradle wrapper & \texttt{gradle/wrapper/gradle-wrapper.properties} \\
\hline
ProGuard rules & \texttt{app/proguard-rules.pro} \\
\hline
Unit tests & \texttt{app/src/test/java/} \\
\hline
Instrumented tests & \texttt{app/src/androidTest/java/} \\
\hline
\end{longtable}

\chapter{Command Snippets}

\section{ADB Commands}
\index{adb}

\begin{lstlisting}[language=bash]
# List devices
adb devices

# Install APK
adb install app-debug.apk

# Uninstall app
adb uninstall com.example.hotelappref

# View logcat
adb logcat

# Clear logcat
adb logcat -c

# Take screenshot
adb shell screencap /sdcard/screen.png
adb pull /sdcard/screen.png

# Pull database file
adb pull /data/data/com.example.hotelappref/databases/hotel_database ./
\end{lstlisting}

\section{Gradlew Commands}

\begin{lstlisting}[language=bash]
# Clean build
./gradlew clean

# Build debug APK
./gradlew assembleDebug

# Build release APK
./gradlew assembleRelease

# Install debug
./gradlew installDebug

# Run tests
./gradlew test

# Check dependencies
./gradlew dependencies
\end{lstlisting}

\printglossaries
\printindex

\end{document}
